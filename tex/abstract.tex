% abstract.tex:

\begin{abstract}

This thesis advances the field of indirect dark matter detection by developing and optimizing current and next generation balloon-borne experiments designed to explore uncharted \ac{DM} parameter spaces. It addresses both hardware and analysis challenges, presenting work on two complementary missions: \ac{GAPS} and the next-generation \ac{GRAMS}\cite{zeng2025gammarayantimattersurveygramsexperiment, vondoetinchem2010generalantiparticlespectrometergaps, aramaki2020dual}. A primary contribution was the assembly and testing of the GAPS payload, one of the largest scientific balloon payloads of all time, which utilizes a novel exotic-atom technique with custom \ac{Si(Li)} detectors to identify low-energy cosmic antinuclei. \ac{GAPS} aims to conduct background-free searches for antideuterons during its long-duration Antarctic balloon flight, while also providing a precision measurement of the cosmic antiproton spectrum. Looking forward, this research involved the design and validation of a novel noble-liquid detector prototype for the \ac{GRAMS} project. \ac{GRAMS} is a novel project that can simultaneously target both astrophysical observations with MeV gamma rays and an indirect dark matter search with antimatter detection \cite{aramaki2020dual}. The author led the design and testing of the detector system for the \ac{GRAMS} prototype (pGRAMS) and oversees its launch.

The scientific motivation for these efforts is rooted in the unique potential of low-energy \ac{ad} and \ac{aHe3} / \ac{aHe4} detection. The flux of these antinucleus from conventional \ac{CRs} interactions is exceptionally low, while signals from dark matter annihilations or decays are predicted to be significantly enhanced in the sub-GeV/n energy range. This high signal-to-background ratio offers a nearly background-free signature for discovering particle dark matter, providing a powerful and complementary approach to direct detection and collider experiments. While at the same time, optimized detector from these balloon-borne experiments will also be able to measure cosmic antiprotons in sub-GeV/n  energy range to provide supplement measurements to BESS, AMS-02 experiment, which helps understand both potential \ac{DM} interaction as well as supernovae process in our universe.

Since AMS-02 experiment reported potential \ac{aHe3} / \ac{aHe4} events \cite{aguilar2013first, PhysRevD.99.023016}, it has drawn people's attention to model \ac{aHe3} flux from both primary \ac{DM} interactions as well as from secondary background including \ac{ISM} interactions etc \cite{PhysRevD.99.023016, PhysRevD.102.063004, Kachelrie_2020, PhysRevLett.126.101101, PhysRevD.97.103011, PhysRevD.96.083020, Ding_2019}. To validate these models, a prediction of \ac{aHe3} flux that current missions could reach is essential. With geant4 simulation, a series of sensitivity is constructed with particularly \ac{GRAMS} configuration reaching 1.47 $\times$ 10$^{-7}$ [m$^2$ s sr GeV/n]$^{-1}$ for 3 \ac{LDB} flight, and 1.55 $\times$ 10$^{-9}$ [m$^2$ s sr GeV/n]$^{-1}$ / / $3.10\times10^{-10}$ [m$^2$ s sr GeV/n]$^{-1}$ for 2 / 10 years satellite flight \cite{ZENG2025103152}. 

\end{abstract}

