\chapter{GAPS Experiment Apparatus}
\label{chap:GAPS}

\begin{figure}[H]
    \centering
    \includegraphics[width=1\linewidth]{fig/GAPS_instrument_v1.png}
    \caption{Left figure shows \ac{GAPS} payload design while right figure shows the actual payload hanging on flight vehicle at McMurdo station}
    \label{fig:GAPS_antarctic}
\end{figure}

As introduction Section.~\ref{chap:intro:GAPS} mentioned, \ac{GAPS} is an Antarctic balloon mission designed to search for low-energy ($< 0.25 GeV/n$) cosmic-ray antinuclei in the austral summer of 2025. It is designed to precisely measure the flux of low-energy cosmic-ray antideuterons, antiprotons, and antihelium. \ac{GAPS} has being contributing to its science community since early 2000s, with several project-wise concrete milestones:
\begin{itemize}
    \item Year 2002 - Year 2004: First idea \cite{HAILEY2004122} and beam test \cite{Hailey_2006}.
    \item Year 2012: pGAPS balloon flight \cite{MOGNET201424, J_E_Koglin_2008}. 
    \item Year 2018 - Year 2019: Science flight detector fabrication \cite{Rogers_2019, KOZAI2022166820, scotti2019frontendelectronicsgapstracker, PEREZ201812, a5c7b362e95a41df8a83d91fe2100f18, OKAZAKI201820}.
    \item Year 2020 - Year 2025: \ac{GAPS} science flight payload assembly and testing\cite{10168972, Tiberio_2025, Aoyama:2025S0}.
    \item Year 2025 - Year 2026: \ac{GAPS} first \ac{LDB} flight on the day of December 16th 2025 at 5:37 am NZDT, see Fig.~\ref{fig:GAPS_antarctic} and Fig.~\ref{fig:GAPS_launch}.
    \item Year 2026 - : Data analysis and prepare for second flight.
\end{itemize}

\begin{figure}
    \centering
    \includegraphics[width=1\linewidth]{fig/GAPS_launch.png}
    \noindent\raggedleft Left photo shows NASA \ac{LDB} inflating the in preparation for launch. Middle photo shows the exact moment of \ac{GAPS}payload left launch vehicle. Right photo shows the \ac{GAPS}payload ascending to the stratosphere with a clear sky background shot from the ground station.
    \caption{GAPS launch on December 16th 2025 at 5:37 am NZDT from McMurdo station, Antarctica.}
    \label{fig:GAPS_launch}
\end{figure}

As my major contribution for \ac{GAPS} started from year 2021, this thesis will focus mainly on the \ac{GFP} and \ac{GAPS} science flight.
The \ac{GAPS} first science balloon flight design is displayed in Fig.~\ref{fig:GAPS_antarctic}. The major detector instrument is built with two primary subsystems: a \ac{TOF} system for particle timing and a high-resolution tracker (\ac{Si(Li)}) for energy and tracking measurement. The \ac{TOF} system features inner and outer layers of EJ-200 plastic scintillator paddles. The outer \ac{TOF} if made of an horizontal plane above the rest of the detector (named “umbrella”) and of four lateral vertical walls (named “cortina”). The inner \ac{TOF} is a cube that surrounds the tracker system on top, bottom and lateral sides. All scintillators are 6.35 mm thick and 16 cm wide, with a length between 1.1 and 1.8 m. Each paddle is read out with \ac{SiPM}(Hamamatsu S13360-6050VE) on both ends. The tracker systems is made of 1008 \ac{Si(Li)} arranged in 7 planes evenly spaced by 10 cm \cite{PEREZ201812, KOZAI2019162695, KOZAI2022166820, SAFFOLD2021165015, 10168972}.
 A tracker plane is made of 6×6 modules, where each module contains 2×2 detectors. Each detector has a cylindrical shape of $ 10cm$ diameter and $ 2.5mm$ thickness and it is segmented into eight strips of equal area. The required operational temperature of $ \sim -40^\circ C$ is achieved with an \ac{OHP} system \cite{doi:10.1142/S2251171714400042, doi:10.1142/S2251171717400062, OKAZAKI201820, FUKE2023168102}. We have developed a noise model for the \ac{Si(Li)} detector system to optimize the operating temperature and peaking time of the readout electronics \cite{Rogers_2019,ARAMAKI201290}
\begin{equation}
    ENC^2 = \left(2qI_{peak} + \frac{4kT}{R_p}\right)F_i \tau + 4kT\left(R_s + \frac{\Gamma}{g_m}\right)F_v \frac{C_{total}^2}{\tau} + A_f C_{total}^2 F_{vf}
\end{equation}
where $ I_{peak} $ is the leakage current, $ R_p $ is the parallel resistance, $ R_s $ is the series resistance, $ C_{total} $ is the total capacitance, $ g_m $ is the transconductance of the input FET, $ \tau $ is the peaking time, $ T $ is the temperature, $ A_f $ is the 1/f noise coefficient, and $ F_i $, $ F_v $, and $ F_{vf} $ are the shape factors for the respective noise contributions \cite{ARAMAKI201290}. The first term represents the shot noise from the leakage current and thermal noise from the parallel resistance, which increase with longer peaking times. The second term represents the thermal noise from the series resistance and the FET channel noise, which decrease with longer peaking times. The third term represents the 1/f noise, which is independent of peaking time. By minimizing the total noise with respect to peaking time, we find an optimal peaking time of $\sim 4\ \mu s$ at an operating temperature of $-40^\circ C$.


The readout is preformed with a dedicated \ac{ASIC} which provides the specific required dynamic range between $ \sim10 keV$ and $ \sim100MeV$ \cite{9564082, RE2023167617, 10330133}. The resulting energy resolution of the \ac{Si(Li)} detector is better than $ 4keV$ at a measurement around $ 60keV$, fulfilling the requirement for X-ray discrimination \cite{Tiberio_2025}.



 Aside from detector systems, a highly integrated thermal system goes into the center \ac{Si(Li)} tracker and dissipate heat through the radiator panels on one side of the payload, details see Section.~\ref{chap:GAPS:thermal}. While the other side of the payload has 16 of the sun panels to provide required power during the flight as well as balancing the payload \ac{COM} on a mechanical perspective. On top of the payload, we put Iridium and TDRSS antenna on the BOOM to get best communication. At the very bottom we put all our electronics in the \ac{EBay}.

\section{Detectors}
\label{chap:GAPS:detectors}
\begin{figure}
    \centering
    \includegraphics[width=1\linewidth]{fig/SiLi_module_display.png}
    \noindent\raggedright Left figure shows \ac{GAPS} \ac{Si(Li)} module which contains 4 detectors, a custom 32-channel ASIC, and a front-end board. Right figure shows a close view of each detector which are wire-bounded from each 8 channels to the front-end board.
    \caption{GAPS detector module}
    \label{fig:GAPS_detector}
\end{figure}
As mentioned in previous section, \ac{GAPS}uses \ac{Si(Li)} detectors as its main tracker and calorimeter. Each detector is 10 cm in diameter and 2.5 mm thick, segmented into 8 strips. A total of 1008 detectors are arranged in 7 layers to form the tracker system. Each layer contains 36 modules, and each module consists of 4 detectors, as shown in Fig.~\ref{fig:GAPS_detector}. The detectors are fabricated using an novel lithium drifting method to create a thick intrinsic layer with high energy resolution at a relatively high operating temperature of -40 °C. The fabrication process is illustrated in Fig.~\ref{fig:GAPS_detector_fab}. This thesis shows simple steps of how we achieve the high-quality \ac{Si(Li)} detectors. For the detailed description, readers can refer to \cite{KOZAI2019162695}.

\medskip

\begin{figure}
    \centering
    \includegraphics[width=1\linewidth]{fig/GAPS_SiLi_fab.png}
    \caption{GAPS \ac{Si(Li)} detector fabrication procedure \cite{KOZAI2022166820}.}
    \label{fig:GAPS_detector_fab}
\end{figure}

\noindent\textbf{Step 1: Procurement of p-type Si crystal}

\noindent Several studies have showed that the purity and uniformity of the Si crystal are essential for obtaining a uniform and large-area Li-drifted layer \cite{Miyachi1988307, Miyachi19944115, etde_20317602, kashiwagi1990fabrication, MURRAY1966330, FONG1982623, LITOVCHENKO2003408}. It is also desirable to realize as light a boron doping as possible for sufficient and uniform compensation of the boron acceptors by Li ions. We successfully developed a high-purity p-type Si crystal specifically for the \ac{GAPS}\ac{Si(Li)} detectors in collaboration with SUMCO Corporation, Japan. Table.~\ref{tab:Si_spec} lists the specifications of the crystal used for our fabrications. For the raw material of our crystal growth, we employ polycrystalline silicon made from mono-silane. The crystal is grown to be oxygen free using the floating zone method with an axis of \(\langle \textbf{111} \rangle\). Both Si crystals with \(\langle \textbf{111} \rangle\) and \(\langle \textbf{110} \rangle\) orientations are used in previous studies, but \(\langle \textbf{111} \rangle\) is more proven for \ac{Si(Li)} detector fabrication, including for Shimadzu's commercial detectors. There is also one report indicating that it is empirically preferable for Li drift \cite{kashiwagi1990fabrication}. Resistivity of $\sim 1000~\Omega\cdot cm$ corresponds to an acceptor concentration of $N_A \approx 10^{13}~\mathrm{atoms/cm}^3$ \cite{LITOVCHENKO2003408}, which is an order of magnitude lower density than that used in some of previous studies of large-area \ac{Si(Li)} detectors \cite{Miyachi1988307, Miyachi19944115, etde_20317602, kashiwagi1990fabrication}. Substrate with a lower $p$-type acceptor concentration requires fewer Li ions for compensation, thus reducing the temperature and time required in the diffusion process. Reduction of the heating treatment prevents the Si crystal from forming defects. The in-plane non-uniformity of the resistivity of our Si wafer is $\sim 10\%$ based on the measurement of sample wafers. The lifetime of minority carriers is an indicator of crystal defects and contaminants. The lifetime of $ \approx 1ms$ is enough to make a high-quality compensated region, as proven by our in-house development \cite{PEREZ201812}. After procuring 2.5 mm-thick wafers cut from the 10 cm-diameter Si crystal, we remove foreign matter on the surface by organic-solvent cleaning with methanol, xylene, and acetone. We then etch the wafer surface with our etchant (a solution of hydrofluoric acid, nitric acid and acetic acid) for 2 min to remove surface contaminants and mechanical defects. Si oxide on the surface is then removed by immersing the wafer in a solution of hydrofluoric acid for 1 min.

\medskip

\begin{table}
\centering
%\renewcommand{\arraystretch}{1.2}
\begin{tabularx}{\linewidth}{l >{\raggedleft\arraybackslash}X}
\midrule
Fabrication method & Floating zone \\
Type & p \\
Dopant & Boron \\
Crystal orientation & \(\langle \textbf{111} \rangle\) \\
Oxygen concentration & \(< 1 \times 10^{16} \, \text{atoms / cm}^3\) \\
Carbon concentration & \(< 2 \times 10^{16} \, \text{atoms / cm}^3\) \\
Resistivity & \(\sim 1000 \, \Omega\) cm \\
Minority carrier lifetime & \(\sim 1\) ms \\
Diameter & \(\sim 100 \, \text{mm (4 inches)}\) \\
\bottomrule
\end{tabularx}
\caption{Specifications of Si crystal used for \ac{GAPS}Si(Li) detector.}
\label{tab:Si_spec}
\end{table}


\noindent\textbf{Step 2: Li evaporation and diffusion}

\noindent 
A vacuum-based thermal Li evaporator is used to deposit and diffuse Li into high-purity Si wafers to form an \(n^{+}\)-layer. The process involves:
\begin{itemize}
    \item Evaporating Li onto the wafer at 280\(^\circ\)C under vacuum (\(<10^{-4}\) Pa)
    \item Diffusing Li to form a \(\sim 100\ \mu\)m thick \(n^{+}\)-layer
    \item Using a custom heater plate for uniform heating across the large-area wafer
\end{itemize}

\noindent The depth of the \(n^{+}\)-\(p\) junction formed by Li diffusion is given by \cite{LITOVCHENKO2003408}
\begin{equation}
    x_{j}=2\sqrt{Dt}\cdot\text{erfc}^{-1}\left(\frac{N_{a}}{N_{0}}\right)
\end{equation}

\noindent where \(D\) is the diffusion constant of Li in Si, \(t\) is diffusion time, \(N_{a}\) is acceptor concentration (\(\approx 10^{13}\ \text{atoms/cm}^3\)), and \(N_{0}\) is Li surface concentration (\(\approx 10^{16}\ \text{atoms/cm}^3\)).

\noindent The diffusion constant for \(\sim 1000\ \Omega\cdot\text{cm}\) \(p\)-type Si is:
\begin{equation}
    D=6\times 10^{-4}\exp\left(\frac{-0.61q}{k_{\text{B}}T}\right)\ \left[\text{cm}^2/\text{s}\right]
\end{equation}
\noindent where \(q\) is elementary charge, \(k_{\text{B}}\) is Boltzmann constant, and \(T\) is temperature.

\noindent Key advantages of using high-purity Si (\(\sim 1000\ \Omega\cdot\text{cm}\)) include lower temperature (280\(^\circ\)C) and shorter heating time (1-2 min + cooling) compared to previous studies using \(\sim 100\ \Omega\cdot\text{cm}\) crystals (300-400\(^\circ\)C for 5-20 min). The Li oxide layer formed on the wafer surface is removed by chemical etching for 1 min with our etchant. This etching also removes mechanical defects and contaminants from the surface \cite{KOZAI2019162695}.
\medskip

\noindent\textbf{Step 3: Evaporation of n-electrode and top-hat machining}

\noindent The detector's electrical contact is formed by thermally evaporating a dual-layer electrode onto the \(n\)-side of the wafer. This consists of an 18\,nm adhesion layer of nickel, followed by a 120\,nm protective layer of gold, deposited at room temperature under a vacuum pressure of \(<10^{-4}\)\,Pa. To define the active area and confine the subsequent Li drift to the central region, the detector circumference is ground using Ultrasonic Impact Grinding (UIG) to produce a top-hat geometry. This geometry is designed with a 97\,mm inner diameter to maximize the sensitive area, while leaving a 1\,mm thick brim that remains \(p\)-type after Li drift for safe handling. UIG is chosen as a cost-effective method suitable for mass production. After machining, the \(n\)- and \(p\)-side surfaces are protected with Apiezon\textsuperscript{\textregistered} wax, and the ground sidewall is etched for 12\,minutes to remove damaged material and contaminants introduced during grinding. Finally, the wax is removed through a sequence of organic-solvent cleaning steps using methanol, xylene, and acetone.

\medskip

\noindent\textbf{Step 4: Lithium(Li) drift}
\begin{figure}
    \centering
    \includegraphics[width=1\linewidth]{fig/GAPS_detector_leakage_current.jpg}\\
    \noindent\raggedright Profiles of the applied bias voltage (top), heater output (middle), and leakage current (bottom) during the drift of a sample detector. The inset plot in each panel displays each parameter's variation in the first 3 hours of the drift, as the voltage is increased step-wise to the 600 V set-point.
    \caption{Lithium drift for \ac{GAPS} \ac{Si(Li)} detector fabrication \cite{KOZAI2019162695}.}
    \label{fig:Li_drift_stage}
\end{figure}
\noindent A uniform Li-drift is achieved by our special \ac{GAPS} drift apparatus, which has been custom-designed for large-area \ac{Si(Li)} detectors. We found that retaining a thin undrifted layer on the p-side effectively suppresses the leakage current. The radial uniformity of the growth of the drifted region during Li drift is key to realizing this thin, uniform undrifted layer. Li drifting is performed in a custom drift apparatus consisting of an electrically grounded heater plate, a pressure contact for applying a bias voltage, a \ac{RTD}, and a controller. The Si wafer is set on the heater plate with the n-side up and, the p-side connected to the grounded heater plate. The pressure contact and RTD are connected to the n-electrode to apply a positive bias voltage and monitor the wafer's temperature. Then the controller automatically controlled the sequence desceibed below. Fig.~\ref{fig:Li_drift_stage} depicts an example of the bias voltage, heater output, and leakage current during the $\approx110$ hours of the drifting routine. Our sample detector is a 10-cm \ac{Si(Li)} detector. The voltage is increased step-wise, in 100 V intervals every 30 min, to prevent rapid increase of the leakage current. The insets in each panel of Fig.~\ref{fig:Li_drift_stage} display the variations over the first 3 hours of the drift, as the voltage is ramped up to the set-point of 600 V. In the first panel, open arrows indicate the timing of the voltage steps. In the final panel, the leakage current shows a step-like increase of  mA corresponding to each voltage increase. As Li drifts toward the p-side and the depletion layer expands from the n-side, the leakage current gradually increases. The Joule heat generated by the leakage current also increases. Displayed by the middle panel, heater output is automatically decreased to compensate for the Joule heat and keep the wafer at $100^\circ C$. At the end of the drift, the depletion layer approaches the p-side. At this point, the leakage current rapidly increases, the wafer temperature exceeds $100^\circ C$ due to the Joule heat, and the heater output decreases to zero. The bias voltage is automatically turned off, i.e., the Li drift is terminated, either when the leakage current reaches 25 mA or when the heater output becomes zero. The wafer is then allowed to naturally cool to room temperature. \cite{KOZAI2019162695}. Under a bias voltage as high as 600 V, the depletion layer expands slightly toward the p-side beyond the "i-p junction" formed between the drifted and undrifted layer. Thus, despite the steep increase in leakage current, a thin undrifted layer is retained on the p-side of our \ac{Si(Li)} detectors after the drifting process. The drifted depth W grows with drift time t, as \cite{osti_4532160}

\begin{equation}
    W = \sqrt{2V\mu_L T}
\end{equation}

\noindent where $ \mu_L $ is the Li mobility related to the diffusion constant (D) by the Einstein relation 
\begin{equation}
    D = \mu_L k_B T / q
\end{equation}
\noindent with the elementary charge q and the Boltzmann constant $ k_B $. For \ac{GAPS} detector drift parameters, we derive $t\approx100h$ to obtain a drifted depth $W\approx2.2mm$ ($90\%$ of the overall 2.5 mm thickness). This calculated result is comparable to \ac{GAPS}actual drift time displayed in Fig.~\ref{fig:Li_drift_stage}. In the case that the Li drift is unexpectedly terminated before $\sim 100h $, we resume the drift sequence to make sure the qulaity of our detectors. The higher bias voltage or temperature would reduce the fabrication time while at the same time will generate more hole-electron pairs, which attract Li-compensation, hence disturbing the ideal Li distribution, which should only compensate for acceptors \cite{goulding1966semiconductor}. The optimum parameters for bias voltage $600V$ and temperature $100^\circ C$ here were experimentally optimized and determined. Normally after the drift, conventional way of the \ac{Si(Li)} detector fabrication will remove the undrifted layer near the p-side region and add a metal contact such as evaporated gold as a Schottky barrier. For \ac{GAPS}case, we found excessive large leakage current under $-35^\circ C$ with this conventional method. But if we kept the $ 100\mu m$ undrifted layer, the leakage current was significantly suppressed, see Fig.~\ref{fig:leakage_current_comparison}. With conventional way of polishing the p-side region, they are used as window for low-energy X-ray detection \cite{lyman1989x, rossington1991si, cox2005improvement}, such detectors adopted Schottky barrier contacts to minimize the p-side insensitive layers while suppressing bulk leakage currents. \ac{GAPS}aims to detect X-rays with energies higher than 20 keV, the $\sim 100\mu m$ insensitive layer is acceptable.

\begin{figure}
    \centering
    \includegraphics[width=0.8\linewidth]{fig/leakage_current_comparison.jpg}\\
    \noindent\raggedright Under $ -35^\circ C $ condition, for the same detector before and after removing the undrifted layer on p-side, the leakage current shows significant difference.
    \caption{Leakage current comparison between p-side finishing methods}
    \label{fig:leakage_current_comparison}
\end{figure}

\medskip

\noindent\textbf{Step 5: Machining grooves for the guard ring and strips}

\noindent The side surface of the top-hat has the largest area of the exposed i-layer which can be easily contaminated and contribute to the leakage current. Thus \ac{GAPS} machined guard-ring groove to suppress this surface leakage current, preventing it from flowing into the readout electronics. The guard-ring and strip grooves are machined using \ac{UIG} to electrically isolate the detector's active area. A circular groove with \(\sim 300~\mu\)m depth and 1-mm width is cut into the \(n\)-side, penetrating the \(n^{+}\)-layer (\(\sim 100~\mu\)m depth) to separate the central readout electrode from the perimeter guard-ring electrode. When biased, the \(i\)-layer between them depletes, providing high-resistance isolation, reducing leakage current by an order of \(<10^{-2}\) when the guard-ring is grounded, see Fig.~\ref{fig:SiLi_guarded_ring}. Simultaneously, eight equal-area readout strips are defined by grooves of identical dimensions, each electrically isolated under operating bias. The 1-mm groove width represents an optimization between minimizing exposed \(i\)-layer surface (to reduce leakage current) and practical constraints of \ac{UIG} tool durability and uniform etching—narrower grooves risk tool damage and trap bubbles during etching. The guard-ring width is set to 2.5 mm, balancing active area maximization against the need for reliable application of etch-resisting wax during processing.
\begin{figure}
    \centering
    \includegraphics[width=0.8\linewidth]{fig/GAPS_detector_leakage_current_guard_ring.jpg}\\
    \noindent\raggedright I-V characteristics of a sample detector at before machining the guard-ring groove (triangles); after machining the groove, with a floated guard-ring (rectangles); and after grounding the guard-ring (circles).
    \caption{Groove and guarded ring for \ac{GAPS} detector}
    \label{fig:SiLi_guarded_ring}
\end{figure}

\medskip

\noindent\textbf{Step 6: Evaporation of p-electrode}

\noindent The metal contact on the p-side is evaporated in the same manner as the n-electrode that is introduced in Step 3. 
\medskip

\noindent\textbf{Step 7: Etching on side of the top-hat and grooves}

\noindent Following groove machining, a final etching process is performed on the side of the top-hat and the \(n\)-side grooves after protecting the \(n\)-electrodes and \(p\)-electrodes with wax. This step removes the UIG-damaged layer, smoothes surfaces, and eliminates contaminants from all exposed silicon areas. Organic-solvent cleaning afterward—particularly with methanol—produces a lightly \(n\)-type surface on the exposed \(i\)-region, preventing electric breakdown under high bias. Optimization studies determined that two discrete etching steps (10 min followed by 5 min, totaling 15 min) effectively minimize leakage currents and produce glassy, smooth groove surfaces, while additional etchings provide no significant further improvement. This approach balances active area preservation with contamination control, as excessive etching enlarges the exposed \(i\)-layer area and increases leakage risk.

\section{Detection Concept}
\label{chap:GAPS:concept}
\begin{figure}
    \centering
    \includegraphics[width=1\linewidth]{fig/GAPS_concept_with_payload.png}
    \noindent\raggedright Left figure shows \ac{GAPS} payload design while right figure shows the detection concept of \ac{GAPS}. An incoming low-energy antinucleus (antiproton, antideuteron, antihelium) is slowed down by the \ac{TOF} and \ac{Si(Li)} layers until it stops in the tracker volume. It then forms an exotic atom by replacing a shell electron, de-excites by emitting characteristic X-rays, and finally annihilates with the nucleus producing a star of pions and protons.
    \caption{GAPS payload and detection concept}
    \label{fig:GAPS_detection_concept}
\end{figure}

Once an charged particle hit the detector active volume, the \ac{TOF} system will record timing and position information. With an advanced trigger selection, survived low-energy charged particles will be recorded inside \ac{Si(Li)} tracker and captured by all our detector strips which both operate as tracker and calorimeter. For normal standard model \ac{CRs} like proton and helium, they will leave an primary track with limited amount of charged secondary generated. While if there is an antiproton or antideuteron captured, the annihilation vertex will be constructed by adding all the charged secondary to construct an unique detector response. 

Taking the primary goal of \ac{GAPS} experiment antideuteron as example. The key point of the experiment is to identify antideuterons from all other type of charged particles. There are different type of rejection to consider about. 

A large part of the expected events would be cosmic normal charged particles particularlly here deuteron and proton. These type of the particles deposit energy based on their kinetic energy, which will be selected by our \ac{TOF} system. Compared to captured primary antiparticles, these signals will be significantly suppressed by the energy deposition within $ GeV/n $ energy region as the annihilation of antinuclei would be easily over $ 1GeV $.

The other major part of the background would be antiprotons. As measured by several other experiments, we already had a good understanding about the cosmic antiprotons, see Fig.~\ref{fig:cosmic_flux}. These type of events are expected to be detected in the tracker, and also annihilate at the same time, make it hard to distinguish by just suppressing with \ac{TOF} and energy. But for same amount of kinetic energy, antiproton and antideuterons have different capabilities of transpassing detector materials and they will have different penetration depth with same initial kinetic energy. Aside from these, \ac{GAPS} has also developed an special technique called exotic atom method, an incoming low-energy antinucleus (antiproton, antideuteron, antihelium) is slowed down by the \ac{TOF} and \ac{Si(Li)} layers until it stops in the tracker volume. It then forms an exotic atom by replacing a shell electron, de-excites by emitting characteristic X-rays, and finally annihilates with the nucleus producing a star of pions and protons, see Fig.~\ref{fig:GAPS_detection_concept}. Because of one extra antinucleons for antideuteron, it will also generate more secondary charged pions / kaons comparing to single-anti-charged antiprotons.

\section{GAPS Functional Prototype (GFP)}
\label{chap:GAPS:GFP}
To confirm \ac{GAPS} functionality for the full payload, we have proposed and tested a smaller scale assembly called \ac{GAPS} functional prototype (\ac{GFP}). \ac{GFP} uses 3 layers of detector modules with $ 2\times6 $ modules installed on each layer. For \ac{TOF}, instead of full structure with "Umbrella", "Cortina" and "inner Cube", \ac{GFP} constructed two largest panel as "top TOF" and "middle TOF" to trigger and reconstruct incoming charged particles. A samller scale thermal system \ac{OHP} / \ac{MCHP} is also designed and fabricated. The rest of the electronics remain the same as flight ones so we could test \ac{GAPS} concept to an extensive level to prepare readiness of the first science flight during pandemic time. This section discusses the construction, design of the GFP. The performance of each individual sub-system and result including the thermal performance, the study of noise in the \ac{Si(Li)} tracker, the measurement of the energy spectrum of minimum ionizing particles and the reconstruction of muon tracks are discussed in Section.~\ref{chap:GAPSresult:GFP}. 

\begin{figure}
    \centering
    \includegraphics[width=0.9\linewidth]{fig/GFP_system_diagram.png}\\
    \caption{\ac{GFP} \ac{TOF} system diagram}
    \label{fig:GFP_TOF_diagram}
\end{figure}

Since the aim of the \ac{GFP} is to demonstrate a ground-based system with a simplified configuration, Fig.~\ref{fig:GAPS_GFP_config} shows the \ac{GFP} overall setup. Major components are listed below,
\begin{itemize}
    \item TOF: two panels, each consisting of twelve paddles of plastic scintillator, separated by about one meter. \ac{TOF} system diagram is shown in Fig.~\ref{fig:GFP_TOF_diagram}. 
    \item \ac{Si(Li)} Tracker: three layers of two rows of \ac{Si(Li)} detector modules, each row containing six 4-detector \ac{Si(Li)} modules, which is $ \sim10\% $ of the \ac{GAPS} full payload. A flight-representative high-voltage and low-voltage power supply (HV/LVPS) and back-end data acquisition system (DAQ) provide power and readout. Fig.~\ref{fig:GFP_detector} shows the actual \ac{GFP} detector module assembly.
    \item Thermal System: a heat exchanger and 12 loops of oscillating heat pipes, \ac{MCHP} system, with a commercial chiller and liquid nitrogen as the cooling source.
\end{itemize}

The \ac{GFP} construction started at Bates Research and Engineering Center of MIT in early 2021 and all the testing was accomplished in August of 2022. The construction of the \ac{GFP} thermal system comes first, followed by installing the \ac{Si(Li)} detectors onto the OHP frame. After that, the \ac{TOF} system was assembled around the tracker and the full system integration was completed. The full \ac{GFP} system is shown in Fig.~\ref{fig:GAPS_GFP_config}.

\begin{landscape}
\begin{figure}
    \centering
    \includegraphics[width=0.95\linewidth]{fig/GAPS_GFP_config.png}
    \noindent\raggedright Overview of the \ac{GFP} setup: \ac{Si(Li)} tracker, \ac{TOF} and Thermal. Instruments which are not shown in the photo: tracker power system (on the bottom of electronic rack) and tracker DAQ computer.
    \caption{GAPS \ac{GFP} setup at Bates lab}
    \label{fig:GAPS_GFP_config}
\end{figure}
\end{landscape}

\subsection{GFP Time of Flight system}
\label{chap:GAPS:GFP:TOF}
The \ac{GAPS} \ac{TOF} system provides essential particle identification input through its measurement of particle $ \beta $, $ dE/dx $, and trajectory. It also provides the primary trigger and serves as a veto for the tracker. The \ac{TOF} system is required to have a timing resolution of $\leq400\ ps$ and a position resolution of $ \leq 6\ cm/layer $, as well as being as hermetic as possible (recording $>98\%$ of tracks stopping in the tracker).

Fig.~\ref{fig:GFP_TOF_diagram} shows the \ac{TOF} schematic diagram. A high-speed digitizer (or readout board, see Fig.~\ref{fig:GFP_TOF}) is developed that demonstrates the required performance. This allows for separating multiple hits on the same paddle and energy deposition measurements up to 200 MeV, which is crucial to suppress events that annihilate in the outer \ac{TOF} umbrella. The readout board uses the DRS-4 ASIC that samples the SiPM waveforms at 2 GS/s, with a buffer length of 1024 samples. Digital control is done via the Xilinx Zynq 7010 (part of Mars ZX2 digital back end) and data are sent to the \ac{TOF} computer over Gbit Ethernet.

The \ac{TOF} trigger uses a two-level design, with a local trigger (L1) board mounted in a readout box close to the paddles and a single main trigger (L2) board located near the \ac{TOF} computer. The trigger is based on the energy deposited in each paddle (slow-moving antiparticles will deposit more energy than almost all protons and He nuclei) and on the number of paddles hit. The local trigger board uses three levels of discrimination to determine whether a paddle is hit (minimum-ionizing or above) and which paddles have energy depositions consistent with low-energy antiparticles. This information is passed to the main board which makes a global trigger decision and alerts the tracker and \ac{TOF} systems for readout. The anticipated L1 trigger rate is $\sim$50--100 kHz; L2 reduces this to 500 Hz.

Full prototype construction and testing of the \ac{TOF} paddles and SiPM preamps has been completed (Fig.~\ref{fig:GFP_TOF}). For a vertical muon, the SiPMs detect an average of 70 photoelectrons. The measured time resolution is $<300 ps$ (Fig. 2), substantially better than the requirement. The position resolutions of 3.0 cm/4.6 cm in the longitudinal/transverse paddle directions translate into an angular resolution for the overall \ac{TOF} system of $<3^\circ$. This performance is conservative, as low-energy antiparticles deposit significantly more energy than vertical \ac{MIP}.

The \ac{GFP} \ac{TOF} system is designed to provide the capacity of tracking incoming muons by measuring the time and energy of particles passing through the two layers, and a muon trigger to the tracker and suitable information (i.e. event number, trigger type, etc.) to correctly tag events so that tracker and \ac{TOF} data can be merged into a common event. The system consists of two 12-paddle layers, covering an area of $ 
6\ ft \times 6\ ft $, with a separation of about one meter. Since the \ac{GFP} tracker only contains two rows on one layer, the coverage of the \ac{GFP} \ac{TOF} provides a sufficient efficiency for the down-going muons. The top \ac{TOF} panel simulates the flight \ac{TOF} umbrella and the bottom panel simulates the cube. One meter separation between the two panels is same to the flight instrument. The two panels are identical at the \ac{GFP} which saves the integration timing but provides a valuable validation of the \ac{TOF} panel integration. The lower panel is $\sim30\ cm$ above the top layer of the tracker which is similar with the flight design.

\begin{figure}
    \centering
    \includegraphics[width=1\linewidth]{fig/GAPS_GFP_TOF.png}
    \noindent\raggedright Left photo shows actual \ac{TOF} system geometry at Bates lab, with both top and middle panel mounted on a customized aluminum rack. Right photo shows one paddle that construct the \ac{TOF} panel, with \ac{SiPM} mounted on two side to readout the light signal. Histogram shows timing resolution of one of the paddle.
    \caption{GFP \ac{TOF} system configuration.}
    \label{fig:GFP_TOF}
\end{figure}

\subsection{Si(Li) detector and module}
\label{chap:GAPS:GFP:detector}
\begin{figure}
    \centering
    \includegraphics[width=0.8\linewidth]{fig/GAPS_GFP_detector.jpg}
    \caption{ A \ac{GFP} \ac{Si(Li)} detector module (a top aluminized polypropylene window is not shown).}
    \label{fig:GFP_detector}
\end{figure}
Both the \ac{GFP} and \ac{GAPS} flight instrument use the same \ac{Si(Li)} detector and module to build the tracker. The \ac{GAPS} \ac{Si(Li)} detector system must provide the absorption depth, energy resolution, tracking efficiency, and active area necessary for the \ac{GAPS} antiparticle identification technique, all within the significant temperature, power, and cost constraints of an Antarctic long-duration balloon (\ac{LDB}) flight. A custom lithium-drifted silicon (\ac{Si(Li)}) detector fabrication method was successfully developed to satisfy the \ac{GAPS} requirements, see Section.~\ref{chap:GAPS:detectors}. The mass-production of $\sim$1100 detectors for \ac{GAPS} initial flight program was completed in March 2020. In addition, a passivation scheme has been established to ensure long-term stability of detector performance. A thermally-cured polyimide (with a silane adhesion promoter) protects the exposed silicon surfaces from environmental degradation and is robust to thermal and mechanical stress \cite{SAFFOLD2021165015}, as confirmed via thermal cycling tests, accelerated environmental exposure trials, and a multi-year detector performance monitoring program.

The \ac{GAPS} \ac{Si(Li)} detector is 10 cm in diameter and 2.5 mm in thickness, segmented into eight strips and a guard ring. For both the flight and GFP, every four \ac{Si(Li)} detectors are grouped together into one module. Fig.~\ref{fig:GFP_detector} shows the module geometry and its components, which are same between the \ac{GFP} and flight instrument.

An aluminum module frame provides the mechanical, electrical, and thermal interface for the detectors. The frame is $9.5'' \times 9.5''$ and $1/8''$ thick. Four holes with the diameter of $\phi = 9.2\ cm$ are cut out to host 4 \ac{Si(Li)} detectors and a few small holes to interface with connectors on front-end board. Each detector is rigidly mounted into place using several fluorosilicone O-rings and G10 clamps, protecting the detectors from shock load at takeoff and cut-down. The guard ring of the detector directly touches the aluminium frame which makes the guard ring and aluminium frame (as well as the module windows) connect with the analog ground. Two corners of the frame are machined to interface with the thermal system and one of them is used to mount the cooling collar (the odd and even layer uses a different corner), which transfers heat to the OHP system. In addition, two purge fittings are mounted to the aluminium frame in order to maintain a good air conductance between two adjacent modules in the same row. During operation of the GFP, as well as the on-ground testing for the \ac{GAPS} full payload, dry nitrogen gas is flushed inside the modules via these purge fittings in order to protect the \ac{Si(Li)} detectors against the environmental humidity.

On top and bottom of the module, a polypropylene window is placed to seal the detector volume using O-rings. These windows are $0.015''$ thick and the outer surface is aluminized. They provide additional environmental protection while maintaining high transmission efficiency for incoming and outgoing particles (exotic atomic x-rays and charged particles).

To read out the signals from \ac{Si(Li)} detectors, a PCB front-end board (FEB) \cite{Scotti:2019zu} is mounted on the aluminium frame inside the module. The FEB houses a core ASIC chip at center and provides the interface connections between the ASIC chip and back-end electronics via 50-pin ERNI connectors. It is also mounted with one HiRose connector for supplying the \ac{Si(Li)} detectors with the high voltage bias, and assembled with a temperature sensor and electronic calibration system. To reduce the dead materials right above the silicon strips, two thin ($ 3\ \mu m$ diameter) aluminium wires are bonded between the FEB and \ac{Si(Li)} strip surface.

\subsection{Si(Li) detector electronics and DAQ}
\label{chap:GAPS:GFP:DAQ}
The \ac{Si(Li)} detector electronics consist of three main subsystems: ASIC front-end electronics, back-end DAQ, and power system. The \ac{Si(Li)} detector electronics must provide (i) high-resolution X-ray spectroscopy ($<4\ keV $ FWHM with $ 40\ pF$ capacitance and $<10\ nA$ leakage current) over the range $\sim20-100\ keV$, (ii) coarse-resolution spectroscopy ($\sim$10\% FWHM) and tracking over the range $ 1-100\ MeV $, (iii) coincidences of $ <1\ \mu s $ between \ac{TOF} and tracker, and (iv) power consumption of $ <10\ mW/channel $ to meet power constraints for the balloon experiment.

\subsubsection*{Front-end electronics}
A custom ASIC has been designed, prototyped, and validated to meet above readout requirements \cite{9564082}. The \ac{GAPS} ASIC is called SLIDER-32 (32 channels \ac{Si(Li)} Detector Readout) while the prototype used for the \ac{GFP} is pSLIDER-32. The ASIC is fabricated with a $ 180\ nm $ CMOS technology. The design is optimized for a working temperature of $\sim-40^\circ C $ and low power consumption. The core of \ac{GAPS} ASIC electronics is a low-noise analog readout channel, featuring a charge-sensitive amplifier, dynamic signal compression, and a semi-Gaussian filter with eight selectable peaking times ($ 0.25-1.8\ \mu s$). The digital section of the ASIC communicates with the DAQ system through a \ac{SPI}. It configures the ASIC operation and handles event acquisition through a second interface with the DAQ. For both the \ac{GFP} and flight, six FEBs are chained into one row and provide the ASIC with analog voltage, calibration settings, and digital communication with the back-end DAQ, and propagates signals through the row \cite{RE2023167617}.

\subsubsection*{Back-end and DAQ}
The back-end DAQ controls the ASICs and processes their data to send to the computer. There is one Back-end DAQ Module (BDM) per tracker layer, based on heritage designs from pGAPS \cite{MOGNET201424, VONDOETINCHEM201493}, NCT/COSI \cite{tomsick2019comptonspectrometerimager} and GRIPS \cite{Jochen_Greiner_2012}. Each BDM consists of a low voltage power supply board, a system interface board that relays information with the computer and the trigger interface unit, and a readout control board that interfaces with the ASICs. The back-end DAQ also consists of the tracker interface boards (IF boards) at the edge of each tracker layer that provide an interface between a tracker row and BDM.

\subsubsection*{Power system}
We custom made a low-power consumption power system to meet the requirements for the \ac{GAPS} balloon flight. The \ac{GAPS} power system consists of the high voltage power supply (HVPS) which provides the HV bias to the \ac{Si(Li)} modules, and low voltage power supply (LVPS) which provides low voltages to the front-end and back-end electronics. It is equipped with an Ethernet interface in order to properly control the power on and off, handle the voltage and current set-up sequences, and monitor the telemetry data. The redundant Ethernet interface is performed by a controller board based on a MicroController. The MicroController is the main interface of the system with the DAQ system (via Ethernet protocol) and provides control of the HVPS and LVPS boards hosted in the sub-rack. Fig.~\ref{fig:GFP_power_schematic} shows the schematic diagram of the \ac{GAPS} power system. All the components of the power system are selected to make it work at a wide range of operative temperature and low atmospheric pressure.
\begin{figure}
    \centering
    \includegraphics[width=1\linewidth]{fig/GFP_power_schematic.png}
    \caption{Schematic diagram of the \ac{GFP} power system.}
    \label{fig:GFP_power_schematic}
\end{figure}

The \ac{GFP} power system has the same design with the flight, but the crate only hosts three pairs of HV and LV cards while can be extended for ten pairs for flight. Each HV card contains 18 individual channels and each channel can supply the HV bias with the maximum voltage of $ 300\ V $ and an accuracy of $1\ V$. Every 6 channels from one HV card are grouped into a single connector and serve for one \ac{Si(Li)} row, while each HV channel can be individually turned on and off, and operated with a ramp-up rate of $ <5\ V/s $. In addition, each HVPS channel can rate up to $ 6\ \mu A $ leak current including the current from the power system itself, which allows to bias the \ac{Si(Li)} module with the leak current up to $ \sim 3\ \mu A $ at the operation temperature $ \sim-40^\circ C $. The LVPS provides four different voltages to the ASIC front-end and back-end electronics with a resolution of $ 1\ mV $: $ 2.8\ V $ for ASIC analog signal, $ 2.8\ V $ for ASIC digital signal, $ 3.3\ V $ for ASIC calibration electronics, and $ 3.3\ V $ for back-end DAQ via the interface board. The 6 modules in a row share the four LV rails with the total power consumption of $ \sim3\ W $. During the operation, the software is designed to monitor the leak current from HVPS with a precision of $ 1-2\ nA $ per channel, voltages and current draws from LVPS with an accuracy of $ 1\ mV $ and $ 0.7\ mA $, as well as the temperatures of the power cards. These measurements are sent to the flight computer for the system control.

\begin{figure}
    \centering
    \includegraphics[width=1\linewidth]{fig/GFP_module_row.png}
    \caption{GFP \ac{Si(Li)} detector electronics connection in the row configuration.}
    \label{fig:GFP_module_row}
\end{figure}
Fig.~\ref{fig:GFP_module_row} illustrates the front-end and back-end electronics connection for one row at the \ac{GFP} setup. Comparing to the flight instrument, the \ac{GFP} electronics connection only adds two patch boards to adapt the connection between the ERNI connection between the \ac{Si(Li)} module and the interface board (IF board). The back-end DAQ communicates with ASICs in the row scheme through SPI. An IF board is attached at the beginning of the row to connect the \ac{Si(Li)} row and back-end electronics (back-end DAQ and low-voltage supplies). For the \ac{GAPS} flight, the IF board is directly mounted on the row via a flex cable which is fabricated with two ERNI connectors. However, the flex cable is not ready during the GFP. So a patch board and ribbon cable is made as the solution. Six modules in a row are chained with flex-rigid cables which are made of Kapton ribbon cable with ERNI connector at two ends. At the other end of the row, a terminator with the resistance of $ 100 \Omega $ is mounted to terminate the row connection. Each \ac{Si(Li)} module in the row is assigned with a unique physical address, numbering from 0 (beginning) to 5 (end) regarding the distance between the module and the readout board. In addition, a separate HV cable is attached to the row to provides the HV bias to each \ac{Si(Li)} module individually.

\subsection{GFP Thermal System}
\label{chap:GAPS:GFP:thermal}
Due to both \ac{GAPS} project timeline and the pandemic situation, the \ac{GFP} thermal system was designed, fabricated, and tested before the flight system was assembled. Thermal system of \ac{GAPS} is highly integrated with detector system, makes it hard to prepare in parallel. Then the \ac{GFP} thermal system is designed to validate the novel \ac{OHP} cooling technique for \ac{GAPS} \ac{Si(Li)} tracker before the full payload is ready, see Section.~\ref{chap:GAPS:thermal:OHP}. The \ac{GFP} thermal system is designed to be as close as possible to the flight thermal system in order to provide a representative validation of the \ac{OHP} technique. This section discusses the design of the \ac{GFP} thermal system. The concept of \ac{OHP} and \ac{MCHP} will be discussed in Section.~\ref{chap:GAPS:thermal:OHP} while the performance of the \ac{GFP} thermal system is presented in Section.~\ref{chap:GAPSresult:GFP:thermal}.

\begin{landscape}
\begin{figure}
    \centering
    \includegraphics[width=1\linewidth]{fig/GFP_thermal.jpg}
    \noindent\raggedright Dual-phase trifluoromethane (\ac{R23}) filled in the reservoir and the rest of the swaglok tubings. With heat generated from the detector and electronics on the right side, liquid R23 will vaporize and push the flow moves up to return to the top and the radiator on the left. Then the R23 will dessipate heat and become liquid again, drop to the bottom of the radiator due to gravity. This will keep the cycle flowing inside each loop.
    \caption{ The \ac{GFP} thermal system design, with only one \ac{Si(Li)} row shown (left) and the setup (right).}
    \label{fig:GFP_thermal}
\end{figure}

\begin{figure}
    \centering
    \includegraphics[width=1\linewidth]{fig/GFP_thermal_lines.png}
    \noindent\raggedright In this photo, two rows of detector L1C2 and L2C2 are tagged. As the time of this picture was taken, only these two rows of detectors are installed and the result from Section.~\ref{chap:GAPSresult:GFP:thermal}, Fig.~\ref{fig:GFP_thermal_result} are based on this configuration dated back on December 1st, 2021.
    \caption{GFP thermal design with each row of detector tagged.}
    \label{fig:GFP_thermal_line}
\end{figure}
\end{landscape}
\subsubsection*{System requirements}
GFP thermal system required cooling down \ac{Si(Li)} detectors down to required $ -40^\circ C $ while at the same time validate again the performance of this novel \ac{OHP} technique that doesn't have any active compressor to reduce the total weight of the payload. It is important for balloon experiment that we keep the payload weight sum as low as possible to reach higher flight altitude, detail information discussed in Section.~\ref{chap:intro:balloon}. Detailed \ac{OHP} / \ac{MCHP} discussed in Section.~\ref{chap:GAPS:thermal:OHP}. There is no weight requirement for \ac{GFP} since it is on the ground, but the goal of validating the system make the design of the \ac{GFP} thermal system as close to flight thermal system as possible.

\subsubsection*{The \ac{GFP} thermal design}
To accommodate the \ac{GFP} tracker configuration, which has 12 modules per layer, the \ac{GFP} OHP tubes are routed into 12 capillary loops in series, as illustrated in Fig.~\ref{fig:GFP_thermal} and Fig.~\ref{fig:GFP_thermal_line}. To simulate the flight radiator, a scaled down cold plate couples the cooling source and the OHP tubes. Same as flight, the OHP tubes are filled with dual-phase trifluoromethane (R23). The tubes were coupled with the \ac{Si(Li)} modules with the aluminum cooling collars as discussed earlier. These cooling collars are anodized in order to electrically isolate the OHP tubes from the \ac{Si(Li)} modules. In order to increase the thermal efficiency, the contact surfaces of the cooling collars are pasted with flight-representative thermal grease.

By design, the cooling collars for the \ac{GAPS} flight have 4 different lengths: (i) $ 9\ cm $ for the bottom layer since this layer is most open to the environment and anticipated to have a larger thermal leak, (ii) $ 4\ cm $ for the top layer to minimize the separation between the \ac{TOF} cube and tracker, (iii) $ 10\ cm $ to connect the top and second top layer to increase the cooling for the top layer, and (iv) $ 6\ cm $ for other normal layers. The \ac{GFP} instrument uses the first three types of cooling collars which present the worst thermal condition, to simulate the flight scenario to the maximum.

To reduce thermal leaks, the entire \ac{GFP} OHP system is insulated by at least $ 4'' $-thick foam from the environment and the tracker is wrapped by a two-layer plastic bag. A low-temperature commercial chiller or liquid nitrogen is used to provide the cooling source to the cold plate. A PID controls the heaters with the feedback of temperature readouts from sensors attached to the reservoir and OHP tubes.

\section{GAPS flight Thermal System}
\label{chap:GAPS:thermal}

Since mentioned in Section.~\ref{chap:GAPS:detectors}, \ac{Si(Li)} detectors perform best under $\sim-40^\circ C$, naturally require \ac{GAPS} thermal system to cool down the tracker to below $ \sim-40^\circ C $ during the flight, with a designed cooling power $ \sim300\ W $. As the payload shows in Fig.~\ref{fig:GAPS_antarctic} and the thermal design schematic shows in Fig.~\ref{fig:GAPS_flight_thermal}, \ac{GAPS} thermal system is designed and integrated closely with detector cube, with 2 major parts included, \ac{MCHP}/\ac{OHP} system and flight insulation package.

The multi-loop capillary heat pipes (MCHP) or oscillating heat pipes (OHP) system contains similar components as \ac{GFP} OHP thermal system described in Section.~\ref{chap:GAPS:GFP:thermal}, with a larger-sized radiator, reservoir and swaglok tubing, see Section.~\ref{chap:GAPS:thermal:OHP}.

During flight, the payload is exposed to the environment. To avoid potential sun radiation causing thermal overheat issue, a series of styrofoam insulation package is designed and manufactured to make the entire payload thermal condition stable during the \ac{LDB} mission over multi-week time scale, see Section.~\ref{chap:GAPS:thermal:foam}.

\begin{landscape}
\begin{figure}
\centering
\includegraphics[width=0.9\linewidth]{fig/GAPS_thermal_design.jpg}
\noindent\raggedright Conceptual diagram of the \ac{GAPS} instrument. To cool the \ac{Si(Li)} tracker consisting of $ >1000 $ detectors, the heat to be removed is transported to a radiator, which is located outside the \ac{TOF} system and pointed to the anti-solar azimuth direction, and then dissipated to space \cite{FUKE2023168102}.
\caption{GAPS flight instrument thermal design schematic}
\label{fig:GAPS_flight_thermal}
\end{figure}
\end{landscape}

\subsection{MCHP/OHP}
\label{chap:GAPS:thermal:OHP}
The heat pipe is a passive heat transfer device that utilizes evaporation and condensation of a working fluid to transport heat over long distances. It consists of a sealed container divided into three sections: an evaporator where heat is added, an adiabatic section with no heat transfer, and a condenser where heat is rejected. When heat is applied to the evaporator, the working fluid vaporizes, creating a pressure difference that drives vapor flow to the cooler condenser. Condensate is returned to the evaporator via capillary, gravitational, electrostatic, or centrifugal forces. Thermally, the heat pipe functions as a small ``engine'' where thermal energy is partially converted into work to overcome frictional forces within the fluid circulation, enabling more efficient heat transport than pure conduction, though it produces no net external work. Structurally, heat pipes are simple devices made from materials such as aluminum, copper, or stainless steel and operate within the freezing and critical points of the working fluid, requiring careful fluid--material compatibility (e.g., water is incompatible with aluminum). Categories of heat pipes include capillary-driven (conventional heat pipes, thermosyphons, oscillating heat pipes, rotating heat pipes), size-based (micro, flat-plate, vapor chambers), temperature-based (cryogenic, high-temperature), and loop-type systems (loop heat pipes, capillary pumped loops) where vapor and liquid flows are separated \cite{ma_oscillating_2015}.

\begin{figure}
    \centering
    \includegraphics[width=1\linewidth]{fig/GAPS_thermal_concept.png}
    \noindent\raggedright Left plot shows the concept of a \ac{OHP} system from Akachi's patent \cite{akachi_heatpipe_1990}. Right figure shows the \ac{GAPS} thermal design that enbeded the \ac{OHP} system with the entire payload that still shares evaporator(heat genrated from detector region), adiabatic section(top and bottom horizontal sessions of the arrows) and condenser(radiator) area.
    \caption{Conventional heat pipe schematic}
    \label{fig:GAPS_thermal_concept}
\end{figure}

Addtionally to heat pipe, The Oscillating heat pipe(\ac{OHP}) is a heat transfer device that functions via thermally excited oscillating motion induced by the cyclic phase change of an encapsulated working fluid. A typical OHP consists of a train of liquid plugs and vapor bubbles which exist in serpentine-arranged, interconnected capillary tubes or channels, as shown in Fig.~\ref{fig:GAPS_thermal_concept}. The oscillating heat pipe (OHP) is partially filled with a working fluid, and its internal diameter is made sufficiently small so that liquid plugs are separated by vapor bubbles. Unlike conventional heat pipes, the OHP requires no wicking structure and offers high manufacturability. It consists of three sections: an evaporator, an adiabatic zone, and a condenser, typically arranged as a meandering tube or channel that passes through heat-reception (evaporator) and heat-rejection (condenser) areas. During operation, continuous condensation in the condenser and evaporation in the evaporator generate a pulsating, non-equilibrium vapor-pressure field that drives oscillatory fluid motion between adjacent channel sections. This results in a complex flow pattern characterized by oscillatory and circulatory liquid-vapor volumes, enabling both sensible (convective) and latent (phase-change) heat transfer. Although phase-change processes help initiate the oscillating motion, most of the heat is transported by sensible heat transfer; consequently, oscillating single-phase flow and heat transfer play a dominant role in OHP performance \cite{ma_oscillating_2015}.

\begin{figure}
    \centering
    \includegraphics[width=1\linewidth]{fig/GAPS_OHP_concept.jpg}
    \noindent\raggedright \textbf{(a)} For each loop, it forms an complete cycle of \ac{OHP} with evaporator, adiabatic section and condenser. With \ac{GAPS} payload, heat generated from the \ac{Si(Li)} detector region will behave as the evaporator (heating section), the top and bottom horizontal sections of the swaglok tubing are the adiabatic section while the radiator is the condenser. The liquid R23 inside the tubing will vaporize when absorbing heat from the detector region, push the flow moves up to return to the top and the radiator. Then the R23 will dessipate heat through radiator (cooling section) and become liquid again, drop to the bottom of the radiator due to gravity, starting the next cycle again. This will keep the cycle flowing inside each loop. \textbf{(b)} Plot shows temperature of each \ac{MCHP} loop section.
    \caption{\ac{GAPS} \ac{MCHP} concept}
    \label{fig:GAPS_MCHP_concept}
\end{figure}
% TODO: maybe add a OHP CAD model here to show 36 loops
As for \ac{GAPS}, we utilized the concept of \ac{OHP} and push it to limit with a 36-loop combined heat pipe system to provide sufficient cooling power for the entire tracker system, see Fig.~\ref{fig:GAPS_thermal_concept}. For each loop, it forms an complete cycle of \ac{OHP} with evaporator, adiabatic section and condenser. With \ac{GAPS} payload, heat generated from the \ac{Si(Li)} detector region will behave as the evaporator (heating section), the top and bottom horizontal sections of the swaglok tubing are the adiabatic section while the radiator is the condenser. The liquid R23 inside the tubing will vaporize when absorbing heat from the detector region, push the flow moves up to return to the top and the radiator. Then the R23 will dessipate heat through radiator (cooling section) and become liquid again, drop to the bottom of the radiator due to gravity, starting the next cycle again. This will keep the cycle flowing inside each loop, see Fig.~\ref{fig:GAPS_MCHP_concept}. With 36 loops working together, the entire \ac{OHP} system can provide sufficient cooling power to keep the entire tracker thermally stable during the flight without any active compressor so the mass and power consumption of the payload can be significantly reduced. Since this entire system with 36 loops behaves as one large looped heat pipe, we called it multi-loop capillary heat pipe (\ac{MCHP}) system.

\subsection{Flight Insulation package}
\label{chap:GAPS:thermal:foam}

\begin{table}
\label{table:styrofoam}
\centering
\begin{tabular}{p{0.8\textwidth}|p{0.1\textwidth}}
\hline
\textbf{Property and Test Method} & \textbf{Value} \\
\hline
Thermal Resistance per inch, ASTM C518 @ 75°F mean temp., ft²•h•°F/Btu, R-value\textsuperscript{(1)}, min. & 5.0 \\
\hline
Compressive Strength\textsuperscript{(2)}, ASTM D1621, psi, min. & 25 \\
\hline
Water Absorption, ASTM C272, \% by volume, max. & 0.1 \\
\hline
Water Vapor Permeance\textsuperscript{(3)}, ASTM E96, perm, max. & 1.5 \\
\hline
Maximum Use Temperature, °F & 165 \\
\hline
Coefficient of Linear Thermal Expansion, ASTM D696, in/in•°F & 3.5 × 10\textsuperscript{-5} \\
\hline
Flexural Strength, ASTM C203, psi, min. & 50 \\
\hline
Dimensional Stability, ASTM D2126, \% linear change, max. & 2.0 \\
\hline
Surface Burning Characteristics\textsuperscript{(4)}, ASTM E84 & \\
\quad Flame Spread\textsuperscript{(3)}, ASTM E84 & 15 \\
\quad Smoke Development, ASTM E84 & 165 \\
\hline
\end{tabular}

\vspace{0.2cm}
\small
\begin{enumerate}
\setlength\itemsep{0.0em}
\item R means resistance to heat flow. The higher the R-value, the greater the insulating power.
\item Vertical compressive strength is measured at 10 percent deformation or at yield, whichever occurs first. Since Styrofoam\texttrademark\ Extruded Polystyrene Foam Insulations are visco-elastic materials, adequate design safety factors should be used to prevent long-term creep and fatigue deformation.
\end{enumerate}
\caption{U.S. Physical Properties of Styrofoam\texttrademark\ Brand Scoreboard XPS Foam Insulation}
\label{tab:styrofoam_properties}
\end{table}

GAPS flight insulation used Styrofoam Brand XPS Insulation from DUPONT \cite{dupont_styrofoam}. The major advantages is the large resistance value of the material, that with thinner thickness one could make compatible thermal performance, see chart.~\ref{table:styrofoam}. We have developed a series of styrofoam pieces that covers the cold part of OHP system to a minimum $ 4'' $ thickness to hold the system thermally stable, see table.~\ref{table:styrofoam_list}.

One example styrofoam pieces GAPS-RAD-MEC-049\_Foam\_Section\_33 are showed in Fig.~\ref{fig:foam_33}. \textbf{(a)} Shows the design for this styrofoam piece which goes on top of the OHP heat transaction area. Small group of grooves are formed to divide all 36 loops of OHP tubings from tracker to radiator. \textbf{(b)} shows a 3D model of this particular piece. \textbf{(c)} shows the actual piece that I made with styrofoam and router. This design will guarantee minimum heat leak from the tubing by cover all of the cold area. The fabrication of these foams are essential for both ground testing and flight while especially crucial for ground testing since we can't put entire payload into a controlled environment as more detail be discussed in Section.~\ref{chap:GAPS:thermal:GCS}

\begin{figure}
    \centering
    \includegraphics[width=1\linewidth]{fig/GAPS-RAD-MEC-049_Foam_Section_33.png}
    \noindent\raggedright \textbf{(a)} Shows the design for this styrofoam piece which goes on top of the OHP heat transaction area. Small group of grooves are formed to divide all 36 loops of OHP tubings from tracker to radiator. \textbf{(b)} shows a 3D model of this particular piece. \textbf{(c)} shows the actual piece that I made with styrofoam and router. This design will guarantee minimum heat leak from the tubing by cover all of the cold area.
    \caption{GAPS payload styrofoam GAPS-RAD-MEC-049 Foam Section 33 design}
    \label{fig:foam_33}
\end{figure}

\subsection{Ground Cooling System (GCS)}
\label{chap:GAPS:thermal:GCS}

As discussed in Section.~\ref{chap:GAPS:detectors}, \ac{GAPS} detectors gain their peak performance under $ \sim -40^\circ C $, which is designed to work under LDB flight environment with dedicated thermal system. While before flight, the entire payload has to be calibrated to provide sensible measurement. For a small scale experiment, the solution is rather straightforward with a \ac{TVAC} testing that contains all the payload inside a environment-controlled chamber. But \ac{GAPS} payload exceeded $ \sim 6.6\ m \times 4.1\ m \times 4.6\ m $ dimension, which makes it extremely hard to provide nominal \ac{TVAC} testing environment. One of the few possible facilities that could handle this type of task would be the Space Simulation Vacuum Chamber at the NASA’s Neil A. Armstrong Test Facility (ATF) in Sandusky, Ohio, but the cost of using it would be also way beyond budget because \ac{GAPS} would need multiple months to perform the detector calibration for more than 1000 channels on board \cite{nasa_glenn_sec}.

Because of the payload's large surface area and the significant temperature difference between the radiator/heat pipes and room temperatures, heat input from the ambient air is considerable and must be removed. The heat input, $Q_0$, at thermal equilibrium can be estimated as follows \cite{Hideyuki_FUKE20232}

\begin{equation}\label{equ:GCS_heatload}
    Q_0 = \frac{S k \Delta T}{L}{L}
\end{equation}
To roughly estimate $Q_0$, we assume that the \ac{OHP}, the radiator, and the detectors have an average temperature of $-55^{\circ}\mathrm{C}$ ($\Delta T \approx 80^{\circ}\mathrm{C}$ colder than the insulator outer surface which must be close to the ambient air temperature) and are surrounded by extruded polystyrene (EPS) foam with an average thickness, thermal conductivity, and total surface area of $L \approx 0.1~\mathrm{m}$, $k \approx 0.03~\mathrm{W/m~K}$, and $S \approx 40~\mathrm{m}^2$, respectively. Based on these approximate estimations, Eq.~\ref{equ:GCS_heatload} provides $Q_0 \approx 1~\mathrm{kW}$. This $1~\mathrm{kW}$ and \ac{Si(Li)} internal generated heat $Q_1 \approx 0.1~\mathrm{kW}$ must be removed from the radiator.

\begin{figure}
    \centering
    \includegraphics[width=1\linewidth]{fig/GCS_mount.png}
    \noindent\raggedright Left figure shows the schematics of \ac{GAPS} \ac{GCS} mounting on the payload. Right picture shows actual \ac{GAPS} \ac{GCS} mounted on the payload at MIT Bates lab.
    \caption{GAPS \ac{GCS} design and mounting}
    \label{fig:GAPS_GCS_setup}
\end{figure}

To perform the ground testing, the ground cooling system (\ac{GCS}) has being designed and assembled to provide similar thermal condition for flight, see Fig.~\ref{fig:GAPS_GCS_setup}. While at the same time all the detector modules are sealed with aluminum window and purged with nitrogen gas, the low vacuum condition ($ \sim 3\ torr $) throughout the entire payload is not accomplished with \ac{GCS} system. The \ac{GCS} cools the radiator and has a simple design. The payload configuration is not changed for testing, but non-flight items may be attached to the payload. A cold plate is attached to the radiator to cool it. The cold plate is tightened to the radiator with bolts uniformly distributed in the plate so that the two surface planes make good contact. A tube is attached to the cold plate, and a coolant flows through the tube, drawing heat from the plate. Once the radiator is sufficiently cooled, the HPS transfers heat from the detectors to the radiator, bringing the detectors down to operating temperature \cite{Hideyuki_FUKE20232}.

As we move forward with \ac{GCS} cooling down \ac{GAPS}detector during preparation for the first flight, we have made several improvements to the \ac{GCS} design to enhance its performance and reliability. These improvements include optimizing the coolant flow rate, enhancing the thermal insulation around critical components, and implementing extra chillers to ensure consistent temperature control. These enhancements have significantly improved the efficiency of the cooling process, allowing for more stable and precise temperature regulation of the detectors, see Fig.~\ref{fig:GAPS_GCS}

\begin{figure}
    \centering
    \includegraphics[width=1\linewidth]{fig/GAPS_GCS.png}
    \noindent\raggedright Left figure shows schematics of \ac{GAPS} \ac{GCS} with 4 cold panels embed 36 loops of methanol cooling lines meifold. Each 2 panels are cooled down by a \ac{ACC} circulating methanol. Right picture shows actual \ac{GAPS} \ac{GCS} setup at Columbia University nevis lab. Red arrows with letters show the methanol line displayed on the left.
    \caption{GAPS Ground Cooling System}
    \label{fig:GAPS_GCS}
\end{figure}

\section{GAPS Time of Flight System}
\label{chap:GAPS:TOF}
The \ac{TOF} flight instrument contains an inner cube, top umbrella and side cortina, all composed of plastic scintillator (EJ-200, 0.6 cm thick). An inner cube completely surrounds the tracker, providing near $100\%$ hermeticity and covering an area of 15 $m^2$. The upper umbrella is above the inner cube with a gap of at least 90 cm and four \ac{TOF} cortina are separated from the sides of the inner cube by at least 30 cm, providing \ac{TOF} and tracking measurements. A total of 160 scintillator paddles are used, either 16 cm wide and 160/180 cm long or 10 cm wide and 108 cm long (depending on location). Light is detected by silicon photomultipliers (SiPMs, Hamamatsu S13360-6050VE) coupled to each end of the scintillator via optical silicone cookies. A preamp board incorporates six SiPMs, provides bias voltage ($\sim58 \ V$) and high-speed signal amplification, and produces high-gain and low-gain outputs for the waveform digitizer and trigger, respectively.
